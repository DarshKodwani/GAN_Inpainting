\documentclass[aps,prd,twocolumn,amsmath,amssymb,floatfix,nofootinbib,superscriptaddress]{revtex4-1}
\usepackage{graphicx}
\usepackage{dcolumn}
\usepackage{bm}
\usepackage[usenames, dvipsnames]{color}
\usepackage[normalem]{ulem}
%\usepackage{subfig}
\usepackage{sidecap}
\usepackage{subcaption}

\usepackage[colorlinks,bookmarks]{hyperref}
\definecolor{linkblue}{rgb}{0,0,0.8}
\definecolor{linkgreen}{rgb}{0,0.5,0}

\hypersetup{pdfpagemode=UseNone, pdfstartview=FitH, linkcolor=linkblue, %
            citecolor=linkgreen, urlcolor=linkblue}
            
\providecommand{\eprint}[1]{\href{http://arxiv.org/abs/#1}{#1}}
\bibliographystyle{apsrev4-1}
\graphicspath{{Figures/}}

\newcommand{\dk}[1]{\textcolor{blue}{DK: #1}}

\begin{document}

\title{Inpainting CMB maps using generative adverserial networks }

\author{Darsh Kodwani and freinds}
\email{darsh.kodwani@physics.ox.ac.uk}
\affiliation{Department of Physics, University of Oxford, Denys Wilkinson Building, Keble Road, Oxford, OX1 3RH, UK.}

\begin{abstract}
We propose a novel method for inpainting noisy masked CMB maps using GAN's (and maybe Autoencoders and variational autoencoders). 

\end{abstract}

\maketitle

%%%%%%%%%%%%%%%%%%%%%%%%%%%%%%%%%%%%%%%%%%%%%%%%%%%

\section{Introduction}

The CMB has been one of the primary sources of information to understand the origin and evolution of the universe for many decades \cite{}. 
The 

\section{Review of current methods}

Review literature and current methods - maybe point out where they fail.
\section{Introducing GANs}

Introduction to GAN's in an intuitive way and the use case for CMB maps. Maybe describe (variational) autoencoders if we decide to use them. 
\section{Analysis}

See how well GAN's do compared to standard techniques to reproduce the maps and maybe the Cl's.

\section{Conclusion}

The GAN's can reproduce the CMB maps with inpainting to $XXX$ accuracy.

\section*{Acknowledgments}
\noindent

DK acknowledges financial support from ERC Grant No: 693024. 

\bibliography{all_active}

\end{document}